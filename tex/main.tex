\documentclass{article}
\usepackage[utf8]{inputenc}
\usepackage{polski}
\usepackage{geometry}
\usepackage{mathabx}
\usepackage{tikz}
\usepackage{amsmath}
\usepackage{graphicx}
\usepackage{subfig}
\usepackage{amsmath}
\usepackage{hyperref}
\usepackage{pdfpages}
\usepackage{siunitx}
\usepackage{listings}

\geometry{
a4paper,
total={170mm,257mm},
left=20mm,
top=20mm
}
\renewcommand\thesection{}
\title{Metoda elementów skończonych\\ Projekt zaliczeniowyN}
\author{Maciej Trątnowiecki}
\date{AGH, Styczeń 2020}


\begin{document}
    \maketitle
    \section{Problem obliczeniowy}
        Przypadł mi problem transportu ciepła.  \\
        $$-k(x)\frac{d^2u(x)}{dx^2}=0$$
        $$u(2) = 0$$
        $$\frac{du(0)}{dx}+u(0)=20$$
        $$k(x) =\begin{cases}1 \quad dla \quad  x\in[0,1] \\ 2\quad dla\quad x \in (1,2] \end{cases}$$
        Szukaną jest funkcja u spełniająca:
        $$[0,2]\ni x\to u(x)\in\mathbb{R}$$
    \section{Postać wariacyjna}
        Niech $v(2)=0$
        $$\int_0^2-k(x)u''(x)v(x)dx = 0$$
        $$\int_0^2-k(x)u''(x)v(x)dx =-\int_0^1u''(x)v(x)dx-2\int_1^2u''(x)v(x)dx$$
        $$-\int_0^1u''(x)v(x)dx-2\int_1^2u''(x)v(x)dx = -[u'(x)v(x)]_0^1+\int_0^1v'(x)u'(x)dx-2[u'(x)v(x)]_1^2+2\int_1^2v'(x)u'(x)dx$$
        $$-[u'(x)v(x)]_0^1-2[u'(x)v(x)]_1^2=-u'(1)v(1)+u'(0)v(0)-2u'(2)v(2)+2u'(1)v(1) = u'(1)v(1)+u'(0)v(0)=$$
        $$=u'(1)v(1)+v(0)(20-u(0))$$
        $$\iff u'(1)v(1)+v(0)(20-u(0)) +\int_0^1v'(x)u'(x)dx+2\int_1^2v'(x)u'(x)dx=0$$
        
        Otrzymuje sformułowanie wariacyjne:
        $$\begin{cases}
        B(u,v) = u'(1)v(1)-u(0)v(0)+\int_0^1v'(x)u'(x)dx+2\int_1^2v'(x)u'(x)dx\\
        L(v) = -20v(0)\\
        B(u,v) = L(v)\\
        u(2)=0\quad\land\quad v(2)=0
        \end{cases}$$
    \section{Kwadratura}
        $$n=2\quad x_i=\pm\frac{1}{\sqrt{3}}\quad w_i=1$$
        Przesuwamy kwadraturę:
        $$\int_x^{\frac{2}{n}+x}fdx=\int_{\frac{2}{n}}^{\frac{2(k+1)}{n}}fdx=\frac{1}{n}(f(\frac{1}{n\sqrt{3}}+\frac{2k+1}{n})+f(\frac{-1}{n\sqrt{3}}+\frac{2k+1}{n}))$$
    

        
\end{document}